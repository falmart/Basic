\documentclass{beamer}
\begin{document}
\begin{frame}
Para la realización de este proyecto llamado Moogle!, que consiste en un buscador 
de texto inteligente, tuve q organizar el trabajo de forma progresiva y eficiente, 
trazando objetivos q ir cumpliendo paso a paso hasta tenerlo completo. Dicho esto 
solo tenía que buscar por dónde empezar y, dado q es un procesador de texto, pues 
lo más inteligente seria cargar y guardar los textos que queremos procesar.
Para esta primera tarea utilice diccionarios como estructura de datos ya que, dada su 
versatilidad y funcionamiento facilitarían el trabajo, pero antes cargue y guarde la 
ruta de los archivos, retire los signos de puntuación innecesarios y normalice los 
textos.
Luego cree dos diccionarios, uno para guardar cuantas veces se repite cada palabra 
en cada texto y otro las veces q se repite cada palabra en general. Para esto utilice 
una tupla conformada por ambos diccionarios.
Con esto ya tendría las piezas para implementar el modelo vectorial.
Mis primeros pasos en la construcción del modelo vectorial fueron la implementación 
del TF (Term Frecuency) o frecuencia normalizada que consiste en determinar la 
frecuencia de aparición de una palabra en un texto y dividirla por la frecuencia de la 
palabra más frecuente en dicho texto y el IDF (Inverse Document Frecuency) dada por 
el logaritmo del resultado de dividir la cantidad de documentos que estamos 
procesando entre la cantidad de documentos q aparece la palabra deseada.

\end{frame}
\end{document}